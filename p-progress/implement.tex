 
\section{Implementation}
We use Python for our implementation. We have a class {\tt pokerCards} that consists
two subclasses {\tt card} and {\tt deck}. Each {\tt card} has two parameters rank and suit.
Rank is a number in $\{2,3,\cdots,14\}$ (where 14 means Ace) and suit is a number in 
$\{1,2,3,4\}$. The {\tt deck} has methods like pop and shuffle.

We also have another class {\tt handEvaluator} that contains functions to evaluate poker hands.
In particular, the method we use is inspired by [1]. The idea here is to assign scores to sets of
five, six, or seven cards such that the hand with higher rank has a higher score and equally-ranked
hands have the exact score. Specifically, {\tt handEvaluator} function receives the hole cards and  
the board (with three to five cards) as its arguments and assigns a real number in $[0,1]$
to the union of hole cards and the board. If {\tt handEvaluator(h1,b) > handEvaluator(h2,b)},
it means that the rank of best five cards in {\tt h1$\cup$b} is higher than the best five cards
in {\tt h2$\cup$b}. Notice that this doesn't mean the {\tt h1} does not have any chance to win.



table UI shows hand
hand 2player objects 
player stack possible action role cards
deck includes cards shuffle pop peek
cards class

handevaluator
abstract cards