\section{Introduction}\label{intro}
Implementing an automated poker player is challenging since poker involves
elements of uncertainty, randomness, strategic interaction, and game-theoretic
reasoning. Heads-up poker is a form of Texas Hold'em poker that is played 
between two players. Poker 
has a rich story of study among game theorists, mathematicians, 
and economists. Recently, there has been lots  of research into 
developing strong programs for playing poker. In [1] authors introduce
the Bayesian Poker Program (BPP), which uses a Bayesian network to model 
poker hands and the opponent's playing behavior. In [2] an algorithm has been developed
to compute the approximate jam/fold equilibrium strategies in tournaments 
with three players. In a jam/fold strategy, the actions player can take is restricted to 
either folding or going all-in. This strategy known to be 
the near-optimal in the two player tournaments. A heads-up no-limit Texas 
Hold'em poker player (we will define these terms later) called Tartanian, 
has been introduced in [3]. Tartanian
uses a discretized betting model to reduce the size of the strategy space.
It also, benefits from a card abstraction model to decrease the problem size.
In [4] a poker program called Poki has been developed. Poki uses reinforcement 
learning techniques to explore and construct statistical models for each opponent,
and exploit based on the observed patterns. 

In this project, we aim to discuss implementing an 
automated heads-up poker player. Our approach to this problem is breaking
the problem into a sequence of simplified problems. We will describe this hierarchy
of problem complexities in later sections.

The rest of this progress report is organize as follows: in \S 2 we describe the
rules of heads-up poker and the dynamics of the game. In \S 3 we will
describe our main model for the project and 
demonstrate the hierarchy of simplified problems and their complexity.
\S 4 describes our implementation
so far. Finally, a discussion on our roadmap and future work is brought in \S5.
