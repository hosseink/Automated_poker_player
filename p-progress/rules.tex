
\section{Rules}
We are going to build our model based on \emph{Doyle's game} which was used 
in the 2007 Association for the Advancement of Artificial Intelligence (AAAI) 
Computer Poker Competition. The game is played between two players that we 
will call player A and player B. 

\BIT
\item \textbf{Blinds:} Every hand, both players start with 1000 chips. In odd hands (every other 
hand), player A is the \emph{small blind} and contributes 1 chip to the pot, while 
player B is the \emph{big blind} and contributes 2 chip to the pot. In even hands the 
role of two players is reversed. 
\item \textbf{Pre-flop:} Two players are dealt random cards (face down) which are called
\emph{hole cards}. Then the small blind can either \emph{fold} (\ie yield all the chips in
the pot to the other player), \emph{call} (contribute chips to the pot such that the number
of contributed chips from two players are equal), or \emph{raise} (contributing more chips
to the pot than the opponent). Notice that the famous \emph{all-in} action is a especial case
of raising. The betting process goes on until one player stops raising (and folds or simply calls).
\item \textbf{Flop:} Three community cards from the rest of the deck are shown. Starting from
the big blind, the betting process starts over similar to pre-flop. Unlike pre-flop where the
players are only using their hole cards to make actions, here the players are getting more
information about the community cards.
\item \textbf{Turn:} A fourth community card dealt face up. The betting process is similar to
flop.
\item \textbf{River:} A fifth (and last) community card is shown. A final round of bets takes place
similar to flop and turn.
\item \textbf{Show down:} In the event that none of the players fold until the end of river round,
two players make the best combination of five cards out of seven cards (two hole cards and 
five community cards). The player with a better combination wins the pot. In the case of
two equally ranked hands, the pot is split.
\EIT
