\section{Motivation}\label{intro}
In this project, we aim to discuss implementing an automated heads-up poker 
player. Heads-up poker is a form of Texas Hold'em poker that is played 
between two players. At
the beginning of the game, each of the two players are dealt two cards and 
the players take actions according to their cards. As the game goes on, some
other cards from the deck are revealed and the players take actions 
accordingly. Eventually, the player who can make better combinations of
the cards on the table with their cards wins the hand. For more information
please refer to the Wikipedia page on poker.

We will break the problem into a hierarchy of complexity levels. In the simplest 
form, we will work on the case in which we can only take passive actions
(only call or fold, as opposed to bet aggressively), assuming that the other player takes actions
which are randomly drawn from a distribution with given parameters (that depend
on their card and the revealed cards.) In the next steps, we will work on the
trade-off between exploration and exploitation to learn the model parameters 
when we do not know the actual distribution for the other player's actions. 
If time allows, we will move on to the case where we can bet aggressively too.

\section{Model}
Let $C$ denote the set of cards. (Each card can be represented by a number
and a suit.) Let $\mathcal X:\Omega \rightarrow C\times C $ be a random variable 
representing the opponent's cards in a particular hand. At any step $i$,
the opponent takes an action $A_i$ which is a realization of a random variable
coming from a distribution
which depends on her cards $X$, the set of revealed cards $O_i$, and the 
history of the hand $H_i$ (previous actions) so far. For example, in the limit poker 
$A_i=\{\mathrm{no~bet},\mathrm{bet~\$1}\}$, whereas in unlimited poker,
$A_i=[0,\mathrm{the~opponent's~stack}]$. Then when it's our turn to act, we take an 
action based on the observations $A_i$, $O_i$, and $H_i$. (In the simplest case, this 
action can only be calling or folding.) The dynamics of the game only allows
us to see the opponent's cards $X$, if we call her bets in all steps. So, there is
an implicit trade-off between exploring the distribution of our opponent's actions
by calling her bets, and exploiting our acquired knowledge about this distribution.

We can evaluate the performance of our model by the regret function, which
quantifies the gap between our collected rewards (winnings) and the best possible rewards.
In the next steps, we will consider the cases where we can bet aggressively. We can also 
consider a broader family of distributions that can incorporate the dependency of our previous 
actions as well. 
Ultimately, we can evaluate our models for the more complicated case, by playing against
humans or other automated poker players.

\section{Approach}
A baseline for our model can be $\epsilon$-explore policy. An oracle can be the case where 
we can see the opponent's hands. The team members believe that our knowledge from 
CS221 especially reinforcement learning, Q-learning, SARSA, exploration-exploitation,
Monte Carlo and function approximation can be extremely helpful in this problem. We
also think that it is a difficult task to build up the hierarchy of models from simple ones
to more complicated ones and model the human behavior. 

We will mostly use Python for our simulations. There are a family of modules 
in Python that provide an interface to hand calculators and game simulators. 
There are plenty of automated poker players and our goal is to compare the performance
of different reinforcement learning methods with the state of the art algorithms. 